\subsection{Впишем диаграмму в формат $185 \times 232$}

Область номограммы представляет собой четырёхугольник, построенный по точкам
$(0; 0)$, $(0; -b_{max}$, $(1, m_{min})$, $(1, m_{max})$.
Левая и правая его стороны - содержат шкалы $b$ и $m$ соответственно, шкалы $C$ и $k$ находятся внутри четырёхугольника.
Сверху и снизу подписи не требуются, можно оставить по 5 мм.
Слева и справа будут подписи шкал, со значениями от 0 до 500 и от 0 до 321, отступим от левого и правого края по 20 мм.
Таким образом, чертёж номограммы, за исключением подписей шкал, нужно уложить в прямоугольник размером $145 \times 222$

\paragraph{Преобразование формы номограммы}

Обозначим ширину и высоту доступной для построения области через $a$ и $b$.

Вертикальные шкалы - параллельны, а форма номограммы в целом - близка к параллелограмму.
Преобразуем номограмму таким образом, чтобы приблизить её к прямоугольнику.
Это можно сделать с помощью афинного преобразования.

$$
\begin{pmatrix}
	x' \\
	y'
\end{pmatrix}
= M
\begin{pmatrix}
	x \\
	y
\end{pmatrix}
+
\vec{v}.
$$

Где 
$$M = \begin{pmatrix}
	\lambda_x & p_x \\
	p_y & \lambda_y
\end{pmatrix}
$$
a $\vec{v}$ - вектор смещения координат $\{x'_0; y'_0\}$.

Раскрывая уравнение, получим
$$x' = \lambda_x x + p_x y + x'_0,$$
$$y' = \lambda_y y + p_y x + y'_0.$$

Мы хотим расположить самую нижнюю точку - $(0; -b_{max})$ - в нижнем левом углу чертежа, а шкалу $m$ - самую длинную и находящуюся справа - вдоль правой стороны чертежа.
Запишем систему
$$
\left\{
\begin{gathered}
	0 = 0 \lambda_x -b_{max} p_x + x'_0, \\
	0 = -b_{max} \lambda_y + 0 p_y + y'_0, \\
	W = 1 \lambda_x + m_{max} p_x + x'_0, \\
	H = m_{max} \lambda_y + 1 p_y + y'_0, \\
	W = 1 \lambda_x + m_{min} p_x + x'_0,\\
	0 = m_{min} \lambda_y + 1 p_y + y'_0.
\end{gathered}
\right.
$$

Она распадается на две подсистемы
$$
\left\{
\begin{gathered}
	0 = 0 \lambda_x -b_{max} p_x + x'_0, \\
	W = 1 \lambda_x + m_{max} p_x + x'_0, \\
	W = 1 \lambda_x + m_{min} p_x + x'_0.
\end{gathered}
\right.
$$
$$
\left\{
\begin{gathered}
	0 = -b_{max} \lambda_y + 0 p_y + y'_0, \\
	H = m_{max} \lambda_y + 1 p_y + y'_0, \\
	0 = m_{min} \lambda_y + 1 p_y + y'_0.
\end{gathered}
\right.
$$

% И решим её методом Гаусса
$$\vec{x} = \{\lambda_x, p_x, x'_0\}$$

$$
\begin{gathered}
	\left(
		\begin{array}{ccc}
			0 & -b_{max} & 1\\
			1 & m_{max} & 1 \\
			1 & m_{min} & 1
		\end{array}
		\right.
		\left|
		\begin{array}{c}
			0 \\ W \\ W
		\end{array}
	\right)
	\Rightarrow
	\begin{array}{c}
		S_2-=S_3\\
		S_3-=S1
	\end{array}
	\left(
		\begin{array}{ccc}
			0 & -b_{max} & 1\\
			0 & m_{max}+m_{min} & 0 \\
			1 & m_{min}+b_{max} & 0
		\end{array}
		\right.
		\left|
		\begin{array}{c}
			0 \\ 0 \\ W
		\end{array}
	\right)	
	\Rightarrow \\
	\Rightarrow
	\begin{array}{c}
		S_2-=S_1 \\
		S_2/=m_{max}+m_{min}
	\end{array}
	\left(
		\begin{array}{ccc}
			0 & -b_{max} & 1\\
			0 & 1 & 0 \\
			1 & 0 & 0
		\end{array}
		\right.
		\left|
		\begin{array}{c}
			0 \\ 0 \\ W
		\end{array}
	\right)	
	\Rightarrow
	\begin{array}{c}
		S_1+=b_{max}S_2
	\end{array}
	\left(
		\begin{array}{ccc}
			0 & 0 & 1\\
			0 & 1 & 0 \\
			1 & 0 & 0
		\end{array}
		\right.
		\left|
		\begin{array}{c}
			0 \\ 0 \\ W
		\end{array}
	\right)	
\end{gathered}
$$

$$
\left\{
\begin{gathered}
	\lambda_x = W, \\
	p_x = 0, \\
	x'_0 = 0.
\end{gathered}
\right.
$$

$$x' = W x.$$


$$\vec{y} = \{\lambda_y, p_y, y'_0\}$$

$$
\begin{gathered}
	\left(
		\begin{array}{ccc}
			-b_{max} & 0 & 1 \\
			m_{max} & 1 & 1 \\
			m_{min} & 1 & 1
		\end{array}
		\right.
		\left|
		\begin{array}{c}
			0 \\ H \\ 0
		\end{array}
	\right)
	\Rightarrow
	\begin{array}{c}
		S_2-=S_3\\
		S_3-=S_1
	\end{array}
	\left(
		\begin{array}{ccc}
			-b_{max} & 0 & 1 \\
			m_{max} - m_{min} & 0 & 0 \\
			m_{min} + b_{max} & 1 & 0
		\end{array}
		\right.
		\left|
		\begin{array}{c}
			0 \\ H \\ 0
		\end{array}
	\right)
	\Rightarrow\\
	\Rightarrow
	\begin{array}{c}
		S_2/=m_{max} - m_{min}\\
	\end{array}
	\left(
		\begin{array}{ccc}
			-b_{max} & 0 & 1 \\
			1 & 0 & 0 \\
			m_{min} + b_{max} & 1 & 0
		\end{array}
		\right.
		\left|
		\begin{array}{c}
			0 \\ \frac{H}{m_{max} - m_{min}} \\ 0
		\end{array}
	\right)
\end{gathered}
$$
Запишем
$$\lambda_y = \frac{H}{m_{max} - m_{min}};$$
Заменим запись и работаем дальше
$$
\begin{gathered}
	\left(
		\begin{array}{ccc}
			-b_{max} & 0 & 1 \\
			1 & 0 & 0 \\
			m_{min} + b_{max} & 1 & 0
		\end{array}
		\right.
		\left|
		\begin{array}{c}
			0 \\ \lambda_y \\ 0
		\end{array}
	\right)
	\Rightarrow\\
	\Rightarrow
	\begin{array}{c}
		S_1+=b_{max} S_2\\
		S_3-=(m_{min} + b_{max}) S_2
	\end{array}
	\left(
		\begin{array}{ccc}
			0 & 0 & 1 \\
			1 & 0 & 0 \\
			0 & 1 & 0
		\end{array}
		\right.
		\left|
		\begin{array}{c}
			b_{max}\lambda_y \\ 
			\lambda_y \\ 
			-(m_{min} + b_{max})\lambda_y
		\end{array}
	\right)
	\Rightarrow
	\left(
		\begin{array}{ccc}
			1 & 0 & 0 \\
			0 & 1 & 0 \\
			0 & 0 & 1 \\
		\end{array}
		\right.
		\left|
		\begin{array}{c}
			\lambda_y \\ 
			-(m_{min} + b_{max})\lambda_y\\
			b_{max}\lambda_y
		\end{array}
	\right)
\end{gathered}
$$

$$
\left\{
\begin{gathered}
	\lambda_y = \frac{H}{m_{max} - m_{min}},\\
	p_y = -(m_{min} + b_{max})\lambda_y,\\
	y'_0 = b_{max}\lambda_y;
\end{gathered}
\right.
$$

$$y' = \lambda_y y + p_y x + y'_0;$$
$$y' = \lambda_y y -(m_{min} + b_{max})\lambda_y x + b_{max}\lambda_y;$$
$$y' = \lambda_y \left( y - (m_{min} + b_{max}) x + b_{max} \right).$$
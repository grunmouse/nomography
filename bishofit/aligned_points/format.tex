\subsection{Впишем диаграмму в формат $185 \times 232$}

Сверху и снизу подписи не требуются, можно оставить по 5 мм.
Слева и справа будут подписи шкал, со значениями от 0 до 500 и от 0 до 321, отступим от левого и правого края по 20 мм.
Таким образом, чертёж номограммы, за исключением подписей шкал, нужно уложить в прямоугольник размером $145 \times 222$

\paragraph{Преобразование формы номограммы}
Вертикальные шкалы - параллельны, а форма номограммы в целом - близка к параллелограмму.
Преобразуем номограмму таким образом, чтобы приблизить её к прямоугольнику.

Найдём, матрицу соответствующего преобразования.
$$
\left(
	\begin{array}{c}
		x' \\
		y'
	\end{array}
\right) 
= 
\left(
	\begin{array}{cc}
		1 & 0 \\
		p & 1
	\end{array}
\right) 
\times 
\left(
	\begin{array}{c}
		x \\
		y
	\end{array}
\right)
;
\Rightarrow
\left\{
	\begin{array}{l}
		x' = x,\\
		y' = px + y.
	\end{array}
\right. 
$$

Нам нужно привести примерно к одному уровню нижние точки шкал $p$ и $m$. 
Они имеют координаты $(0; -b_{max})$ и $(1; 0)$ соответственно
Так как наша матрица не смещает точки, лежащие на оси ординат, будем смещать точку $(1; 0)$ до ординаты $-b_{max}$
$$-b_{max} = p \cdot 1 + 0;$$
$$p = -b_{max}.$$

После такого преобразования матрица сильно сместилась вниз, нужно сместить её обратно, прибавлением вектора $\{0; b_{max}\}$.
Тогда:
$$
\left\{
	\begin{array}{l}
		x' = x,\\
		y' = -b_{max}x + y + b_{max}.
	\end{array}
\right. 
$$

\paragraph{Размер номограммы и модули шкал}

Общая высота номограммы равна $m_{max}$.
Значит вертикальный масштаб номограммы
$$\lambda_y = \frac{222}{m_{max}}.$$

Общая ширина номограммы равна 1.
$$\lambda_x = 145.$$
\subparagraph{}Исходя из свойств номограммы из выравненных точек и связи пределов $b_{max}$, $k_{max}$ и $m_{max}$ соответствующие концы шкал должны лежать на одной прямой.
Чтобы проверить это, найдём псевдоскалярное произведение
$$
\begin{gathered}
	\left(\vec{r}(k_{max}) - \vec{r}(b_{max}) \right) \wedge \left(\vec{r}(m_{max}) - \vec{r}(b_{max}) \right)
	=
	\left|
		\begin{array}{cc}
			x(k_{max}) - x(b_{max}) & y(k_{max}) - y(b_{max}) \\
			x(m_{max}) - x(b_{max}) & y(m_{max}) - y(b_{max})
		\end{array}
	\right|
	= \\
	=
	\left|
		\begin{array}{cc}
			\frac{h k_{max}}{1+h k_{max}} - 0 & 0 + b_{max} \\
			1 - 0 & m_{max} + b_{max}
		\end{array}
	\right|
	= 
	\begin{vmatrix}
		\frac{h k_{max}}{1+h k_{max}} & b_{max} \\
		1 & m_{max} + b_{max}
	\end{vmatrix}
	= \frac{h k_{max}}{1+h k_{max}} \cdot \left(m_{max} + b_{max} \right) - b_{max} 
	= \\
	= \frac{h k_{max}}{1+h k_{max}} \cdot \left(m_{max} + h k_{max} m_{max}\right) - h k_{max} m_{max}
	= \frac{h k_{max}}{1+h k_{max}} \cdot m_{max} \left(1 + h k_{max}\right) - h k_{max} m_{max}
	= \\
	= h k_{max} m_{max}  - h k_{max} m_{max} = 0.
\end{gathered}
$$
Действительно, точки лежат на одной прямой.
Эта прямая может быть задана уравнением
$$
\begin{vmatrix}
	x & y + b_{max} \\
	1 & m_{max} + b_{max}
\end{vmatrix}
=0.
$$
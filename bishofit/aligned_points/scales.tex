\subsection{Уравнения шкал}

$$k: \left\{
\begin{array}{l}
	x = \frac{hk}{1+hk},\\
	y = 0.
\end{array}
\right.
$$

$$m: \left\{
\begin{array}{l}
	x = 1,\\
	y = m.
\end{array}
\right.
$$

$$b: \left\{
\begin{array}{l}
	x = 0,\\
	y = -b.
\end{array}
\right.
$$

$$
C: \left\{
\begin{array}{l}
	x = \frac{1}{2}, \\
	y = \frac{C}{2}.
\end{array}
\right.
$$

\paragraph{Рассчитаем координаты концов каждой шкалы}
\subparagraph{k:}
$$x(k_{min}) = \frac{hk_{min}}{1+hk_{min}} = \frac{0}{1+0} = 0;$$
$$x(k_{max}) = \frac{h k_{max}}{1+h k_{max}} = \frac{2.136 \cdot 0.3}{1 + 2.136 \cdot 0.3} = 0.3905;$$
$$y(k) = 0.$$

\subparagraph{m:}
$$x(m) = 1;$$
$$y(m_{min}) = m_{min} = 100;$$
$$y(m_{max}) = m_{max} = 500.$$

\subparagraph{b:}
$$x(b) = 0;$$
$$y(b_{min}) = -b_{min} = 0;$$
$$y(b_{max}) = -b_{max} = -321.$$

\subparagraph{C:}
$$x(C) = 0.5;$$
$$y(C_{min}) = \frac{C_{min}}{2} = \frac{35}{2} = 17.5;$$
$$y(C_{max}) = \frac{C_{max}}{2} = \frac{500}{2} = 250.$$

\paragraph{Геометрический анализ расположения концов шкал}
\subparagraph{}Пределы $k_{min}$ и $b_{min}$ - всегда нулевые. Следовательно координаты соответствующих им точек шкал совпадают и равны $(0; 0)$.
\subparagraph{}Предел $C_{max} = m_{max}$, следовательно
$$y(C_{max}) = \frac{C_{max}}{2} =\frac{y(m_{max})}{2}$$
С учётом того, что $x(C) = 0.5 x(m)$, точки $C_{max}$ и $m_{max}$ лежат на прямой, проходяешей через 0.
$$\frac{y(C_{max})}{x(C_{max})} = \frac{y(m_{max})}{x(m_{max})} = y(m_{max}).$$
Эта прямая ограничивает номограмму сверху, т.к. ни одна шкала не заходит за неё.

\subparagraph{}Снизу номограмма ограничена прямой, проходящей через точки $b_{max}$ и $m_{min}$. Эти точки не связаны непосредственно. Чтобы проверить, не заходят ли за неё шкалы, найдём уравнение этой прямой
$$
\begin{vmatrix}
	x - x(b_{max}) & y - y(b_{max}) \\
	x(m_{min}) - x(b_{max}) & y(m_{min}) - y(b_{max})
\end{vmatrix} 
= 0;$$
$$\begin{vmatrix}
	x & y + b_{max} \\
	1 & m_{min} + b_{max}
\end{vmatrix} = 0;$$
$$x (m_{min} + b_{max}) - y - b_{max} = 0.$$

Введя теперь функцию 
$$f(x, y) = x (m_{min} + b_{max}) - y - b_{max};$$
И подставив в неё координаты концов шкал, мы должны получить значения одного знака.
$$f(0, 0) = - b_{max};$$
Начало координат, служащее так же началом шкалы $k$ даёт отрицательное значение. Значит остальные точки должны дать неположительное.
$$
\begin{gathered}
f\left(x(k_{max}), 0\right) 
	= x(k_{max}) (m_{min} + b_{max}) - b_{max}
	= \frac{h k_{max}}{1+h k_{max}} \left(m_{min} + h k_{max} m_{max} \right) - h k_{max} m_{max}
	= \\
	= h k_{max} \left( \frac{m_{min} + h k_{max} m_{max}}{1+h k_{max}} - m_{max} \right)
	= h k_{max} \left( \frac{m_{min} + h k_{max} m_{max} - m_{max} - h k_{max} m_{max} }{1+h k_{max}}  \right)
	= \\
	= h k_{max} \left( \frac{m_{min} - m_{max} }{1+h k_{max}}  \right).
\end{gathered}
$$
$$h k_{max} >0; 1+h k_{max} > 0; m_{min} - m_{max} <0;$$
Следовательно
$$f\left(x(k_{max}\right), 0) < 0$$

$$
\begin{gathered}
f\left(x(C_{min}),y(C_{min})\right) 
	= x(C_{min}) (m_{min} + b_{max}) - y(C_{min}) - b_{max}
	= 0.5 (m_{min} + b_{max}) - \frac{C_{min}}{2} - b_{max}
	= \\
	= \frac{m_{min} + b_{max} - C_{min}}{2} - b_{max}
	= \frac{m_{min} + h k_{max} m_{max} - m_{min} \left( 1 - hk_{max} \right )}{2} - h k_{max} m_{max}
	= \\
	= \frac{m_{min} + h k_{max} m_{max} - m_{min} + hk_{max}m_{min}}{2} - h k_{max} m_{max}
	= h k_{max}\left(\frac{m_{max} + m_{min}}{2} - m_{max} \right)
	= \\
	= h k_{max}\left(\frac{m_{min} - m_{max}}{2} \right).
\end{gathered}
$$
$$f\left(x(C_{min}),y(C_{min})\right) <0.$$
Проверено. Ни одна из шкал не выдаётся за пределы чётырёхухольника построенного по точкам
$(0; 0)$, $(0; -b_{max}$, $(1, m_{min}$, $(1, m_{max})$.

\subsection{Вторая формула}
\paragraph{Найдём каноничную форму $\pi(b, C, m)$}

$$m = b + C;$$
$$b + C - m = 0;$$
$$
\begin{array}{l}
x = b,\\
y = C;\\
\end{array}
\Rightarrow
\left\{
	\begin{array}{l}
		x + y - m = 0 \\
		x - b = 0,\\
		y - C = 0.\\
	\end{array}
\right.
$$

\paragraph{$\pi(b, C, m)$}
$$
\left|
\begin{array}{lll}
	1 & 1 & -m \\
	1 & 0 & -b \\
	0 & 1 & -C
\end{array}
\right|
=0.
$$
Прибавим к первому столбцу второй
$$
\left|
\begin{array}{lll}
	2 & 1 & -m \\
	1 & 0 & -b \\
	1 & 1 & -C
\end{array}
\right|
=0.
$$
$$S_1 /= 2$$
$$
\left|
\begin{array}{lll}
	1 & 0.5 & -0.5m \\
	1 & 0 & -b \\
	1 & 1 & -C
\end{array}
\right|
=0.
$$
$$R_1 \leftrightarrow R_3$$
$$
\left|
\begin{array}{lll}
	-0.5m & 0.5 & 1 \\
	-b & 0 & 1 \\
	-C & 1 & 1
\end{array}
\right|
=0. (\pi')
$$

\section{Формулы связи количества воды и бишофита с концентрацией раствора}

\subsection{Переменные и общие формулы}
\paragraph{} Пусть 
$A$ - масса $MgCl_2$, 
$B$ - масса связанной воды в бишофите, 
$C$ - масса жидкой воды, используемой для растворения.

\paragraph{}$B+C$ - масса воды в растворе.
\paragraph{}$m=A+B+C$ - общая масса раствора.
\paragraph{}$k = \frac{A}{m}$ - массовая концентрация раствора
\paragraph{}$b = A+B$ = масса бишофита.
\paragraph{}$d = \frac{A}{B}$ - соотношение компонентов бишофита 

\paragraph{Выведем формулу}
$$
\left\{
\begin{array}{l}
B = \frac{A}{d},\\
A =  km,\\
b = A + B;
\end{array}
\right.
\Rightarrow
b = km \left(1 + \frac{1}{d} \right ).
$$

Обозначим $$ h = 1 + \frac{1}{d}$$

$$b = h k m.$$

$$\frac{b}{m} = h k.$$

$$C = m - b.$$

$$C = m (1 - h k).$$ 

\subsection{Расчёт констант}
\paragraph{Бишофит:} $MgCl_2 \cdot 6H_2O$.

$$d = \frac{A}{B} = \frac{M\left( MgCl_2 \right)}{6 \cdot M\left( H_20 \right)}$$
$$A_r\left( O \right) = 16.00 \pm 0.005;$$
$$A_r\left( H \right) = 1.01 \pm 0.005;$$
$$A_r\left( Mg \right) = 24.31 \pm 0.005;$$
$$A_r\left( Cl \right) = 35.45 \pm 0.005;$$

$$M\left( MgCl_2 \right) = 24.31 + 2 \cdot 35.45 = 95.21 \pm 0.015 \frac{g}{mol};$$
$$M\left( H_2O \right) = 2 \cdot 1.01 + 16 = 18.02 \pm 0.015 \frac{g}{mol};$$
$$M\left(6 \cdot H_2O \right) = 6 \cdot \left(18.02 \pm 0.015 \right) = 108.12 \pm 0.09 \frac{g}{mol};$$

$$d = \frac{A}{B} = \frac{M\left( MgCl_2 \right)}{M\left(6 \cdot H_2O \right)}.$$
$$\Delta_{d} 
	= \Delta_{M\left( MgCl_2 \right)} + \Delta_{M\left( 6 \cdot H_2O \right)} 
	= \frac{0.015}{95.21} + \frac{0.09}{108.12} 
	\approx 0.00099
;$$
$$d 
	= \frac{95.21}{108.12} \cdot \left( 1 \pm \Delta_d \right )
	= 0.8806 \pm 0.0009.
$$

$$h= 1 + \frac{1}{d}.$$
$$\Delta_\frac{1}{d} = \Delta_d;$$

$$h 
	= 1 + \frac{1}{d} \cdot \left( 1 \pm \Delta_d \right )
	= 1 + 1.136 \pm 0.002
	= 2.136 \pm 0.002.
$$

\subsection{Принимаемые значения}
$$0 < k \le 0.3;$$
$$100 \le m \le 500;$$
$$0 < b \le 150h \Rightarrow 0 < b \le 321;$$
$$35 \le C \le 500$$
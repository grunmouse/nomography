
\section{Формулы связи количества воды и бишофита с концентрацией раствора}

\paragraph{} Пусть 
$A$ - масса $MgCl_2$, 
$B$ - масса связанной воды в бишлфите, 
$C$ - масса жидкой воды, используемой для растворения.

\paragraph{}$B+C$ - масса воды в растворе.
\paragraph{}$m=A+B+C$ - общая масса раствора.
\paragraph{}$k = \frac{A}{m}$ - массовая концентрация раствора
\paragraph{}$b = A+B$ = масса бишофита.
\paragraph{}$d = \frac{A}{B}$ - соотношение компонентов бишофита 

\paragraph{Бишофит:} $MgCl_2 \cdot 6H_2O$.

$$d = \frac{A}{B} = \frac{M\left( MgCl_2 \right)}{6 \cdot M\left( H_20 \right)}$$

\paragraph{Молярная масса}
$$M\left( MgCl_2 \right) = 24.5 + 2 \cdot 35.5 = 95.3 \frac{g}{mol}$$
$$M\left( H_20 \right) = 2 \cdot 1 + 16 = 18 \frac{g}{mol}$$
$$d = \frac{95.3}{6 \cdot 18} = 0.88$$

\paragraph{Выведем формулу}
$$
\left\{
\begin{array}{l}
B = \frac{A}{d},\\
A =  km,\\
b = A + B;
\end{array}
\right.
\Rightarrow
b = km \left(1 + \frac{1}{d} \right ).
$$

Обозначим $$ h= 1 + \frac{1}{d}$$

$$b = k m h.$$

$$\frac{b}{m} = k h.$$

$$C = m - b; \Rightarrow \frac{C}{m} = 1 - \frac{b}{m}.$$

\section{Анализ формул, подготовка номограммы}

\paragraph{Даны формулы}
$$b = k m h;$$
$$C = m - b.$$
Где $b$ - масса бишофита, $C$ - масса жидкой воды, $k$ - массовая доля $MgCl_2$ в растворе, $m$ - общая масса, $h$ - константа.

\paragraph{Найдём каноничную форму $\pi(b, C, k)$}
$$m = b + C,$$
$$b = kh \left( b+ C \right);$$
$$kh b + kh C = b;$$
$$\left(kh -1\right) b + kh C = 0.$$

$$
\begin{array}{l}
x = b,\\
y = C;\\
\end{array}
\Rightarrow
\left\{
	\begin{array}{l}
		\left(kh -1\right) x + kh y = 0 \\
		x - b = 0,\\
		y - C = 0.\\
	\end{array}
\right.
$$

\paragraph{$\pi(b, C, k)$}
$$
\left|
\begin{array}{lll}
	kh -1 & hk & 0 \\
	1 & 0 & -b \\
	0 & 1 & -C
\end{array}
\right|
=0.
$$
Вычтем из первого столбца второй

$$
\left|
\begin{array}{lll}
	kh -1 - hk & hk & 0 \\
	1     - 0  & 0  & -b \\
	0     - 1  & 1  & -C
\end{array}
\right|
=0;
\Rightarrow
\left|
\begin{array}{lll}
	-1 & hk & 0 \\
	1  & 0  & -b \\
	-1 & 1  & -C
\end{array}
\right|
=0;
$$
Умножим первую и третью строки на -1
$$
\left|
\begin{array}{lll}
	1 & -hk & 0 \\
	1 & 0   & -b \\
	1 & -1  & C
\end{array}
\right|
=0;
$$
Поменяем местами первый и третий столбцы
$$
\left|
\begin{array}{lll}
	0  & -hk & 1 \\
	-b  & 0   & 1 \\
	C & -1  & 1 
\end{array}
\right|
=0;
$$
Получена каноничная форма $\pi'$.
\paragraph{Найдём уравнения шкал}
$$k:
\begin{array}{l}
	x = 0,\\
	y = -hk;
\end{array}
$$
$$b:
\begin{array}{l}
	x = -b,\\
	y = 0;
\end{array}
$$
$$C:
\begin{array}{l}
	x = C,\\
	y = -1;
\end{array}
$$

\paragraph{Найдём каноничную форму $\pi(b, C, m)$}

$$m = b + C;$$
$$b + C - m = 0;$$
$$
\begin{array}{l}
x = b,\\
y = C;\\
\end{array}
\Rightarrow
\left\{
	\begin{array}{l}
		x + y - m = 0 \\
		x - b = 0,\\
		y - C = 0.\\
	\end{array}
\right.
$$

\paragraph{$\pi(b, C, m)$}
$$
\left|
\begin{array}{lll}
	1 & 1 & -m \\
	1 & 0 & -b \\
	0 & 1 & -C
\end{array}
\right|
=0.
$$
Прибавим к первому столбцу второй
$$
\left|
\begin{array}{lll}
	2 & 1 & -m \\
	1 & 0 & -b \\
	1 & 1 & -C
\end{array}
\right|
=0.
$$


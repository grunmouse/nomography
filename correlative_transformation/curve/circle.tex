\subsection{Окружность}
$$s: \: 
\left\{ \begin{gathered}
	x = R\sin{t}+a,\\
	y = R\cos{t}+b.
\end{gathered} \right.
$$

$$f = R\sin{t}+a; \; g = R\cos{t}+b;$$
$$f' = R\cos{t}; \; g' = -R\sin{t}.$$

$$\bar{s}:
\:
\left\{
\begin{gathered}
	\bar{x} =  -\frac{g'}{\Delta},\\
	\bar{y} =  \frac{f'}{\Delta}.
\end{gathered}
\right.
$$

$$\Delta = f g' - g f' = (R\sin{t}+a)(-R\sin{t}) - (R\cos{t}+b) R\cos{t} $$
$$\Delta = -\left(R^2\sin^2{t}+aR\sin{t} + R^2\cos^2{t} +bR\cos{t}\right);$$
$$-\Delta = R^2 + aR\sin{t} + bR\cos{t};$$
$$-\frac{\Delta}{R} = R + a\sin{t} + b\cos{t};$$
Обозначим 
$$D = -\frac{\Delta}{R}.$$

\begin{multicols}{2}
\subparagraph{}
$$\bar{x} =  -\frac{g'}{\Delta};$$
$$\bar{x} =  \frac{R \sin{t}}{\Delta};$$
$$\bar{x} =  -\frac{\sin{t}}{D};$$

\subparagraph{}
$$\bar{y} =  \frac{f'}{\Delta};$$
$$\bar{y} =  \frac{R \cos{t}}{\Delta};$$
$$\bar{y} =  -\frac{\cos{t}}{D};$$
\end{multicols}

\subparagraph{} Введём полярную систему координат
$$\left\{
\begin{gathered}
	\bar{\rho} = \frac{\sqrt{(g')^2 + (f')^2}}{\abs{\Delta}},\\
	\bar{\varphi} = \sign{f'}\left(\frac{\pi}{2}\sign{\Delta} + \phi \right).
\end{gathered}
\right.
$$

$$\sqrt{(g')^2 + (f')^2} =  \sqrt{(R\cos{t})^2 + (R\sin{t})^2} = \sqrt{R^2\left( (\cos{t})^2 + (\sin{t})^2 \right)} = \abs{R};$$
$$\bar{\rho} = \frac{\abs{R}}{\abs{\Delta}};$$
$$\bar{\rho} = \frac{1}{\abs{D}}.$$

$$\bar{\varphi} = \sign\brkt{R\cos{t}}\left(\frac{\pi}{2}\sign{\Delta} + \phi \right).$$

$$\sign{\Delta} = -\sign{D}.$$


$$\sign{y} = -\sign\brkt{\cos{t}}\sign{D};$$
$$\frac{\bar{x}}{\bar{\rho}} = -\sign{D}\sin{t};$$
$$\bar{\varphi} = -\sign\brkt{\cos{t}}\sign{D}\acos\brkt{-\sign{D}\sin{t}};$$

$$\psi = -\frac{\pi}{2}-t;$$
$$\sin{\psi} = -\cos{t};$$
$$\cos{\psi} = -\sin{t}.$$

$$D = R - a\cos{\psi} - b\sin{\psi},$$
$$\bar{\rho} = \frac{1}{\abs{D}},$$
$$\bar{\varphi} = \sign\brkt{\sin{\psi}}\sign{D}\acos\brkt{\sign{D}\cos{\psi}}.$$

Обозначим: $c = \sqrt{a^2 + b^2}$;
$$D = R - c\left( \frac{a}{c}\cos{\psi} + \frac{b}{c}\sin{\psi} \right);$$
Введём такой угол $\theta$
$$\cos{\theta} = \frac{a}{c},\; \sin{\theta} = \frac{b}{c};$$
$c$ и $\theta$ - полярные координаты центра окружности.
$$D = R - c\left( \cos{\theta}\cos{\psi} + \sin{\theta}\sin{\psi} \right);$$
$$D = R - c\cos\brkt{\psi - \theta};$$

Функция $\varphi(t)$ оказывается кусочно заданной для $D>0$ и $D<0$, а при $D=0$ должен происходить предельный переход.

\begin{minipage}{0.4\textwidth}
$$D>0$$
$$\bar{\rho} = \frac{1}{D},$$
$$\bar{\varphi} = \sign\brkt{\sin{\psi}}\acos\brkt{\cos{\psi}};$$
$$\bar{\varphi} = \psi;$$
$$\bar{\rho} = \frac{1}{R - c\cos\brkt{\bar{\varphi} - \theta}};$$
$$\bar{\rho} = \frac{\frac{1}{R}}{1 - \frac{c}{R}\cos\brkt{\bar{\varphi} - \theta}}.$$
Это уравнение конического сечения с фокусом в точке O.
\end{minipage}
\begin{minipage}{0.4\textwidth}
$$D<0$$
$$\bar{\rho} = -\frac{1}{D},$$
$$\bar{\varphi} = \sign\brkt{\cos{t}}\acos\brkt{\sin{t}};$$
$$\sin{\varphi} = \cos(t),\; \cos{\varphi} = \sin{t};$$
$$\bar{\varphi} = \frac{\pi}{2} - t = \psi + \pi;$$
$$\psi = \bar{\varphi} - \pi;$$
$$\bar{\rho} = -\frac{1}{R - c\cos\brkt{\bar{\varphi} - \pi - \theta}};$$
$$\bar{\rho} = -\frac{\frac{1}{R}}{1 + \frac{c}{R}\cos\brkt{\bar{\varphi}  - \theta}}.$$
Это уравнение левой ветви гиперболы.
\end{minipage}

\paragraph{Анализ свойств $D$}
$$D = R - c\cos\brkt{\psi - \theta};$$
Если $R>c$, то $D>0$. Если $R \le c$, то $D$ имеет переходы через 0 в точках
$$\cos\brkt{\psi - \theta} = \frac{R}{c}.$$
\begin{minipage}{0.4\textwidth}
С положительной стороны:
$$D>0;$$
$$R - c\cos\brkt{\psi - \theta} > 0;$$
$$R > c\cos\brkt{\psi - \theta};$$
$$\cos\brkt{\bar{\varphi} - \theta} < \frac{R}{c};$$
$$\acos{\frac{R}{c}} < \brkt{\bar{\varphi} - \theta} < 2\pi - \acos{\frac{R}{c}}.$$
\end{minipage}
\begin{minipage}{0.4\textwidth}
С отрицательной стороны:
$$D<0;$$
$$R - c\cos\brkt{\psi - \theta} < 0;$$
$$\cos\brkt{\bar{\varphi} - \pi - \theta} > \frac{R}{c};$$
$$-\cos\brkt{\bar{\varphi} - \theta} > \frac{R}{c};$$
$$\cos\brkt{\bar{\varphi} - \theta} < -\frac{R}{c};$$
$$\acos\brkt{-\frac{R}{c}} < \brkt{\bar{\varphi} - \theta} < 2\pi - \acos\brkt{-\frac{R}{c}}.$$
\end{minipage}

Иными словами, зависимость $D(\varphi)$ неоднозначна, и выражается двумя функциями, с несколько разными областями определения.

\paragraph{}Найдём соответствующие значения $t$:
$$\cos\brkt{\psi - \theta} = \frac{R}{c},$$
$$\psi = -\frac{\pi}{2}-t;$$
$$\cos\brkt{-\frac{\pi}{2} - t - \theta} = \frac{R}{c};$$
$$\cos\brkt{\frac{\pi}{2} + t + \theta} = \frac{R}{c};$$
$$\sin\brkt{t + \theta} = \frac{R}{c};$$
$$t + \theta = \frac{\pi}{2} \pm \acos\frac{R}{c};$$
$D<0$, при $\asin\frac{R}{c} < t + \theta < \pi - \asin\frac{R}{c},$
$D>0$, при $\pi - \asin\frac{R}{c} < t + \theta < 2\pi - \asin\frac{R}{c}.$

$\asin\frac{R}{c}$ - это полуугол развала касательных к окружности $r$, проходящих, через начало координат. Поскольку прямая, проходящая через начало координат, отображается в несобственную точку, логично, что касательная, проходящая через начало координат, отображается в несобственную точку касания, а соответствующая ей точка касания - в асимптоту гиперболы или ось параболы.
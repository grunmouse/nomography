\subsection{Три точки, лежащие на одной прямой}
\paragraph{}Даны три различные точки $P$, $Q$, $R$, лежащие на одной прямой:
$$\begin{vmatrix}
	P_x & P_y & 1 \\
	Q_x & Q_y & 1 \\
	R_x & R_y & 1
\end{vmatrix}
= 0.$$
Они преобразуются в прямые
$$\bar{p}:\: P_x \bar{x} + P_y \bar{y} + 1 = 0;$$
$$\bar{q}:\: Q_x \bar{x} + Q_y \bar{y} + 1 = 0;$$
$$\bar{r}:\: R_x \bar{x} + R_y \bar{y} + 1 = 0.$$
Т.к. система строк матрицы
$$\begin{pmatrix}
	P_x & P_y & 1 \\
	Q_x & Q_y & 1 \\
	R_x & R_y & 1
\end{pmatrix}$$
линейно зависима, а никакая из подсистем её строк - не линейно зависима, то эта система разрешима относительно $(\bar{x}; \bar{y})$.
\subsection{Пучок прямых}
\paragraph{}Пусть существует пучок прямых $\mathbf{P}$, с центром в точке $P$ с координатами $(a; b)$.
$$\mathbf{P}:\: y-b = k(x - a);$$
Каждую прямую этого пучка назовём $p_k$, где $k$ - значение параметра.
$$\frac{y}{ka-b} - \frac{kx}{ka-b} + 1 = 0;$$
\paragraph{}Каждая прямая $p_k$ отображается в пучок прямых $\mathbf{\bar{P}}_k$, с центром в точке $\bar{P}_k$ c координатами:
$$\bar{x}(k) = -\frac{k}{ka-b},$$
$$\bar{y}(k) = \frac{1}{ka-b}.$$
Эти точки лежат на одной прямой, если $\frac{d\bar{y}}{d\bar{x}}$ не зависит от $k$.
$$\frac{d\bar{x}}{dk} = - \frac{ka-b-ka}{\left(ka-b\right)^2} =  \frac{b}{\left(ka-b\right)^2};$$
$$\frac{d\bar{y}}{dk} = - \frac{a}{\left(ka-b\right)^2};$$
$$\frac{d\bar{y}}{d\bar{x}} = -\frac{a}{b}.$$
Теперь можно найти саму прямую:
$$\bar{x}(0) = 0,$$
$$\bar{y}(0) = -\frac{1}{b};$$
$$\bar{y} - \bar{y}(0) = \frac{d\bar{y}}{d\bar{x}}\left(\bar{x} - \bar{x}(0)\right);$$
$$\bar{y} + \frac{1}{b} = -\frac{a}{b}\bar{x};$$
$$a\bar{x} + b\bar{y} + 1 = 0.$$
Это та же самая прямая, в которую отображается точка $P$.
\paragraph{Следовательно:} Точку на плоскости можно интерпретировать как пучок прямых, пересекающихся в этой точке, а пучок прямых, можно интерпретировать как точку, являющуюся его центром.
\paragraph{} Можно говорить, что точка отображается в прямую, а прямая - в точку, причём, закон отображения - симметричен, относительно плоскостей.
\subsection{Отображение кривой}
Кривая $q$, представляет собой множество точек $Q_t$, каждая из которых должна отобразиться в прямую. Таким образом, получается семейство прямых $\bar{\mathbf{Q}}$, в общем случае, не имеющих общей точки пересечения.
\paragraph{Пусть}
$$q:\: 
\left\{
\begin{array}{l}
	x = f(t),\\
	y = g(t);
\end{array}
\right.
$$
где функции $f(t)$, $g(t)$ гладкие. Обозначим их производные по $t$: $\displaystyle f' = \frac{\partial f}{\partial t}$, $\displaystyle g' = \frac{\partial g}{\partial t}$.
$$\mathbf{\bar{Q}}:
\:
f(t)\bar{x} + g(t)\bar{y} + 1 = 0.$$
\subsubsection{Огибающая $\bar{q}$ семейства прямых $\mathbf{\bar{Q}}$}
\paragraph{Введём функцию:}
$h(\bar{x}, \bar{y}, t) = f(t)\bar{x} + g(t)\bar{y} + 1.$

Найдём её частные производные
$$
\begin{aligned}
	\frac{\partial h}{\partial t} &= f'\bar{x} + g'\bar{y};
	&
	\frac{\partial h}{\partial\bar{x}} &= f;
	&
	\frac{\partial h}{\partial\bar{y}} &= g;
	\\
	\frac{\partial^2 h}{\partial t^2} &= f''\bar{x} + g''\bar{y};
	&
	\frac{\partial^2 h}{\partial t \partial\bar{x}} &= f';
	&
	\frac{\partial^2 h}{\partial t \partial\bar{y}} &= g'.
\end{aligned}
$$

\paragraph{Параметрические уравнения огибающей определяются системой уравнений:}
[http://www.math24.ru/огибающая-семейства-кривых.html]
$$
\left\{
\begin{gathered}
	h = 0, \\
	\frac{\partial h}{\partial t} = 0;
\end{gathered}
\right.
$$
при дополнительном условии
$$\frac{\partial^2 h}{\partial t^2} \ne 0.$$
$$
\frac{\partial^2 h}{\partial t^2} \ne 0;
\Rightarrow
f''\bar{x} + g''\bar{y} \ne 0;
\Rightarrow
\left[
\begin{gathered}
	f'' \ne 0, \\
	g'' \ne 0;
\end{gathered}
\right.
$$
Это условие выполняется, если $q$ - не прямая.

$$
\left\{
\begin{gathered}
	h = 0, \\
	\frac{\partial h}{\partial t} = 0;
\end{gathered}
\right.
\Rightarrow
\left\{
\begin{gathered}
	f\bar{x} + g\bar{y} + 1 = 0, \\
	f'\bar{x} + g'\bar{y} = 0;
\end{gathered}
\right.
\Rightarrow
\begin{pmatrix}
	f & g \\
	f' & g'
\end{pmatrix}
\begin{pmatrix}
	\bar{x} \\
	\bar{y}
\end{pmatrix}
=
\begin{pmatrix}
	-1 \\
	0
\end{pmatrix}.
$$
Применим метод Крамера:
$$\Delta = 
\begin{vmatrix}
	f & g \\
	f' & g'
\end{vmatrix}
= f g' - g f';
$$
$$\Delta_{\bar{x}} = 
\begin{vmatrix}
	-1 & g \\
	0 & g'
\end{vmatrix}
= - g';
\;
\Delta_{\bar{y}} = 
\begin{vmatrix}
	f & -1 \\
	f' & 0
\end{vmatrix}
= f';
$$
$$
\left\{
\begin{gathered}
	\bar{x} = \frac{\Delta_{\bar{x}}}{\Delta},\\
	\bar{y} = \frac{\Delta_{\bar{y}}}{\Delta};
\end{gathered}
\right.
\Rightarrow
\left\{
\begin{gathered}
	\bar{x} = -\frac{g'}{\Delta},\\
	\bar{y} = \frac{f'}{\Delta}.
\end{gathered}
\right.
$$

\subsubsection{Точка касания}
\paragraph{Пусть} дана прямая $\bar{q}_s$, в которую отображается точка $Q_s (x(s); y(s))$. Найдём параметр $t$ точки её касания с кривой $\bar{q}$.
$$x(s) = f(s);\; y(s)= g(s);$$
$$\bar{q}_s: \: f(s)\bar{x} + g(s)\bar{y} + 1 = 0;$$
$$-f(s)\frac{g'}{\Delta} + g(s)\frac{f'}{\Delta} + 1 = 0;$$
$$\frac{g(s) f' - f(s) g'}{f g' - g f'} = -1;$$
$$f(s) g' - g(s) f' = f g' - g f';$$
$$(f(s) - f) g' = (g(s) - g) f';$$
$$\left[
\begin{gathered}
	t = s,\\
	\frac{f(s) - f}{g(s) - g} = \frac{f'}{g'};
\end{gathered}
\right.
$$
Решение $t = s$ - существует всегда.

\subsubsection{Касательная к $q$}
Назовём касательную в точке $Q_t$ прямой $q_t$ (по аналогии с $\bar{q}_s$).
$$q_t:\: \frac{x - f}{f'} = \frac{y - g}{g'};$$
$$g'x - g'f - f'y + f'g = 0;$$
$$\frac{g'}{f'g - g'f}x - \frac{f'}{f'g - g'f}y + 1 = 0.$$
Эта прямая отображается в точку 
$$
\left\{
\begin{gathered}
	\bar{x} = \frac{g'}{f'g - g'f}, \\
	\bar{y} = - \frac{f'}{f'g - g'f};
\end{gathered}
\right.
\Rightarrow
\left\{
\begin{gathered}
	\bar{x} = - \frac{g'}{g'f - f'g}, \\
	\bar{y} = \frac{f'}{g'f - f'g}.
\end{gathered}
\right.
$$
Легко видеть, что это та же самая точка $\bar{Q}_t = \bar{q}(t)$.

\paragraph{Следовательно:} Если кривая отображается в семейство прямых, имеющих общую огибающую, то касательная к этой кривой в некоторой точке и точка касания отображаются соответственно в точку на огибающей и касательную к ней в этой точке.

\subsubsection{Полярная система координат:}
$$\left\{
\begin{gathered}
	\bar{x} = \bar{\rho}\cos{\bar{\varphi}}, \\
	\bar{y} = \bar{\rho}\sin{\bar{\varphi}}.
\end{gathered}
\right.
$$

$$\bar{\rho}^2 = \bar{x}^2 + \bar{y}^2;$$
$$\bar{\rho}^2 = \frac{(g')^2 + (f')^2}{\Delta^2};$$
$$\bar{\rho} = \frac{\sqrt{(g')^2 + (f')^2}}{\abs{\Delta}}.$$

$$\bar{\varphi} = \sign{\bar{y}}\acos{\frac{\bar{x}}{\bar{\rho}}};$$
$$\sign{\bar{y}} = \sign{\frac{f'}{\Delta}} = \sign{f'}\sign{\Delta};$$

$$\frac{\bar{x}}{\bar{\rho}} = \frac{
	(-g')\abs{\Delta}
}{
	\Delta\sqrt{(g')^2 + (f')^2}
} = -\frac{g'\sign{\Delta}}{\sqrt{(g')^2 + (f')^2}};$$
$$\bar{\varphi} = \sign{f'}\sign{\Delta}\acos\brkt{-\frac{g'\sign{\Delta}}{\sqrt{(g')^2 + (f')^2}}};$$
$$\bar{\varphi} = \sign{f'}\sign{\Delta}\left(\frac{\pi}{2} - \asin\brkt{-\frac{g'\sign{\Delta}}{\sqrt{(g')^2 + (f')^2}}}\right);$$
$$\bar{\varphi} = \sign{f'}\sign{\Delta}\left(\frac{\pi}{2} + \sign{\Delta}\asin{\frac{g'}{\sqrt{(g')^2 + (f')^2}}}\right);$$
$$\bar{\varphi} = \sign{f'}\left(\frac{\pi}{2}\sign{\Delta} + \asin{\frac{g'}{\sqrt{(g')^2 + (f')^2}}}\right).$$

Введём угол $\phi$;
$$\sin\phi = \frac{g'}{\sqrt{(g')^2 + (f')^2}};$$
$$\bar{\varphi} = \sign{f'}\left(\frac{\pi}{2}\sign{\Delta} + \phi \right).$$

$$\left\{
\begin{gathered}
	\bar{\rho} = \frac{\sqrt{(g')^2 + (f')^2}}{\abs{\Delta}}\\
	\bar{\varphi} = \sign{f'}\left(\frac{\pi}{2}\sign{\Delta} + \phi \right).
\end{gathered}
\right.
$$
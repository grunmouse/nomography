\subsection{Преобразование прямой}
\paragraph{} Дана прямая 
$$m: \: ax + by + 1 = 0.$$
По закону преобразования, она должна отобразиться во множество прямых:
$$\mathbf{\bar{M}}: \:
\left\{
	\begin{array}{l}
		x\bar{x} + y\bar{y} + 1 = 0,\\
		ax + by + 1 = 0.
	\end{array}
\right.
$$

$$y = -\frac{ax+1}{b};$$
$$x\bar{x} -\frac{ax+1}{b}\bar{y} + 1 = 0;$$
$$bx\bar{x} -ax\bar{y} - \bar{y} + b = 0;$$
$$x \left(b\bar{x} - a\bar{y} \right) - \bar{y} + b = 0.$$
Найдём такие $(\bar{x}; \bar{y})$, чтобы уравнение было верно для любого $x$. Это возможно только в том случае, если $x$ исключается из уравнения.

$$
\left\{
	\begin{array}{l}
		b\bar{x} - a\bar{y} = 0,\\
		- \bar{y} + b = 0;
	\end{array}
\right.
\Rightarrow
\left\{
	\begin{array}{l}
		\bar{x} = a,\\
		\bar{y} = b;
	\end{array}
\right.
$$
Следовательно, все прямые семейства $\mathbf{\bar{M}}$ пересекаются в точке с координатами $(\bar{x}; \bar{y}) = (a; b)$. Назовём эту точку $\bar{M}$. Т.е $\mathbf{\bar{M}}$ - пучок прямых, а точка $\bar{M}$ - его центр.
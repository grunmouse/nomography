\section{Определение}


\subsection{Обозначения}
\subparagraph{Исходная плоскость} $xOy$.
\subparagraph{Двойственная плоскость} $\bar{x}\bar{O}\bar{y}$, всё принадлежащее ей, будет обозначаться с надчёркиванием, координаты в ней так же будем обозначать $(\bar{x}; \bar{y})$.
\subparagraph{Точки} обозначим заглавными буквами $N$, $\bar{N}$.
\subparagraph{Прямые} обозначим строчными буквами $n$, $\bar{n}$.
\subparagraph{Семейства прямых} обозначим жирными заглавными буквами $\mathbf{N}$, $\mathbf{\bar{N}}$.

\subsection{Закон преобразования:}
\paragraph{}Каждой точке $N$ с координатами $(x; y)$ соответствует прямая $\bar{n}$ задаваемая уравнением $x\bar{x} + y\bar{y} + 1 = 0$.
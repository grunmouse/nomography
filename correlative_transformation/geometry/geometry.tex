
\section{Обозначения}
\paragraph{Операторы}
\subparagraph{Отображение} $\mapsto$
$$A\mapsto a$$
\subparagraph{Принадлежит} $\in$
$$A \in a$$
\subparagraph{Имеют общую точку} (пересекаются или касаются) $\cap$
$$a \cap b = C$$
\subparagraph{Пересекаются} $\sqcap$
$$a \sqcap b = C$$
\subparagraph{Касаются} $\overline\cap$
$$a \overline\cap b = C$$
\subparagraph{Порядок кривой} $'$ 
$$a' = 1$$
$$b' \ne 1$$

\paragraph{Сущности}
\subparagraph{Точки} обозначим заглавными буквами $N$, $\bar{N}$.
\subparagraph{Прямые} обозначим строчными буквами $n$, $\bar{n}$.
\subparagraph{Семейства прямых} обозначим жирными заглавными буквами $\mathbf{N}$, $\mathbf{\bar{N}}$.

\paragraph{Системы координат и выделенные сущности}
\subparagraph{Исходная плоскость} $xOy$.
\subparagraph{Двойственная плоскость} $\bar{x}\bar{O}\bar{y}$. Любой элемент в двойственной плоскости обозначается с надчёркиванием.
\subparagraph{Координаты} $x, y$; $\bar{x}, \bar{y}$.
\subparagraph{Полярные координаты} $\rho, \varphi$; $\bar\rho, \bar\varphi$
\subparagraph{Начало координат} $O$, $\bar{O}$.
\subparagraph{Несобственная прямая} (геометрическое место несобственных точек) $o$, $\bar{o}$.


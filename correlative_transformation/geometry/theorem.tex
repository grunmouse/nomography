\subsection{Аксиомы}
$$\forall A \; \exists \bar a^1:\: A \mapsto \bar a^1$$

\subsection{Теоремы, доказанные выше}
%$$\forall A \; \exists \bathbr{A}

\paragraph{} Точка $A$ отображается в прямую $\bar{a}$.
\paragraph{} Точка $A$ может интерпретироваться как пучок прямых $\mathbf{A}$, с центром в этой точке.
\paragraph{} Закон отображения - симмертричен. Повторное отображение даёт исходный объект.
\paragraph{} Пучок параллельных прямых может интерпретироваться как несобственная точка
\paragraph{} Прямая $a$ отображается в точку $\bar{A}$.
\paragraph{} Прямая, проходящая через начало координат отображается в несобственную точку.
\paragraph{} Начало координат отображается в несобственную прямую.
\paragraph{} Кривая может интерпретироваться как геометрическое место множества точек или как огибающая множества прямых.
\paragraph{} Кривая отображается в кривую, каждая касательная к которой является отображением точки исходнной кривой.
\paragraph{} Если кривая $a \mapsto \bar{a}$, а прямая $b$ касается $a$ в точке $C$, то прямая $\bar{c}$ касается $\bar{a}$ в точке $\bar{B}$.
\paragraph{} Если кривая $a$ $O \in a$, то кривая $\bar{a}$ имеет точку на бесконечности (касание с несобственной прямой).
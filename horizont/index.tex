\section{Вывод рабочей формулы}

Пусть треугольник $OAB$ касательной к поверхности Земли и двумя радиусами. $O$ - центр Земли, $AB$ - касательная, $B$ - точка касания.
\begin{eqnarray}
OB = R;\\
OA = R+h;\\
\angle ABO = 90^\circ;
\end{eqnarray}
\begin{align*}
R & - \text{радиус Земли};\\
h & - \text{высота точки A над Землёй}.
\end{align*}

По теореме Пифагора:
\begin{equation}
AB^2 = OA^2 - OB^2.
\end{equation}

Обозначим: $AB = d$ - расстояние до горизонта при наблюдении из точки $A$.

$$d^2 = (R+h)^2 - R^2;$$
\begin{equation}
	d^2 = 2Rh + h^2.
\end{equation}

Достроим треугольник $OBC$, образованный тем же радиусом $OB$, той же касательной $AB$, продолжающейся в $BC$, и дополнительным радиусом, проходящим через точку $C$.

\begin{eqnarray*}
OC = R+h_c;\\
BC = d_c.
\end{eqnarray*}

\begin{equation}
	d_c^2 = 2Rh_c + h_c^2.
\end{equation}

Обозначим: $AC = l$ - расстояние между точками $A$ и $B$, предельная дальность видимости точки точки с высотой $h_c$ из точки с высотой $h$.

\begin{equation}
	l = \sqrt{2Rh + h^2} + \sqrt{2Rh_c + h_c^2}.
\end{equation}

Функции шкал:
\begin{eqnarray}
	f_l = l;\\
	f_h = \sqrt{2Rh + h^2};\\
	f_{h_c} = \sqrt{2Rh_c + h_c^2}.
\end{eqnarray}

$$f_l - f_h - f_{h_c} = 0.$$

\section{Разработка номограммы с подвижными шкалами}

Уравнение представимо линейкой с одним движком.

Уравнение нужно решать относительно $l$ или относительно $h_c$. Переменная $h$ меняется редко. 

Расположим шкалу $f_l$ на одной линейке, а $f_h$ и $f_{h_c}$ на движке
Тогда линейка будет реализовывать уравнение:
$$x_l - x'_h + x'_{h_c} = 0.$$

Тогда:
$$x_l = m f_l;$$
$$x'_h = \pm m f_h;$$
$$x'_{h_c} = \mp m f_{h_c}.$$

Шкалы $f_h$ и $f_{h_c}$ - противонаправлены. Целесообразно разместить их на одном носителе с общим нулём. 

Тогда расстояние: 
$$x'_{h_c} - x'_h = \pm m \left( f_{h_c} + f_h \right) = \pm m f_l = \pm x_l.$$

Это позволяет совмещать одну из высот с нулём дальности, тогда нулевая высота совместится с расстоянием до горизонта, а вторая высота - с предельной дальностью видимости. 

Для шкалы $f_h$ выбрано обратное направление шкалы, потому что это позволяет совмещать с нулём шкалы $f_l$ значение $h$.

\section{Пределы шкалы}

Пределы шкалы рассчитаем в общем виде через длину и модуль шкал. Пусть длина шкалы - $\Delta x_{max}$, модуль - $m$. Тогда:
$$l_{max} = \frac{\Delta x_{max}}{m}.$$

$$d^2 = 2Rh + h^2;$$
$$h^2 - 2Rh - d^2 = 0;$$
$$\frac{D}{4} = R^2 + d^2;$$
$$h(d) = \sqrt{R^2 + d^2} - R.$$

$$d(h) = \sqrt(2Rh + h^2).$$

$$d = \frac{\Delta x}{m}.$$

$$h(\Delta x) = \sqrt{R^2 + \left(\frac{\Delta x}{m}\right)^2} - R.$$
$$\Delta x(h) = m \sqrt{2Rh + h^2}.$$

Общая длина шкалы высоты:
$$\Delta x_{max} = \Delta x(h) + \Delta x(h_c).$$

Если произвольно ограничить $h = h_{max}$, то
$$\Delta x(h_{c,max}) = \Delta x_{max} - \Delta x(h_{max});$$
$$h_{c,max} = h\left(\Delta x_{max} - \Delta x(h_{max})\right).$$
